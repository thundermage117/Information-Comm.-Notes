\documentclass{article}
\usepackage{mathrsfs}
\usepackage{amsmath}


\begin{document}
\begin{center}
\textbf{\huge{Week 6}}
\end{center}

\section{Single random variable source coding}

Let $\underbar{c}(x)$ be the codeword assigned to $x \in \mathcal{X}$.

$l(x)$ be the length of codeword assigned to $x$.

Here we are coding a single random variable and all codewords are binary strings. (Fixed-variable length source coding)
$$ \mathscr{C}= \{ \underbar{c}(x): x \in \mathcal{X}\}$$

$$ L_{\mathscr{C}}= \sum_{x \in \mathcal{X}}p(x)l(x)$$

Our goal is to design a code which has minimum $L_{\mathscr{C}}$, $L^{*}$
$$L^{*} = \text{min}_{\mathscr{C}} L_{\mathscr{C}}$$

\subsection{Kraft inequality}

Let $\mathscr{C}$ be any prefix-free (binary) code. Then,
$$ \sum_{x \in \mathcal{X}} 2^{-l(x)} \leq 1$$

Proof: We know that any prefix-free code can be represented using a binary tree which has `leaves' as the codeword. In the binary tree corresponding to the code, the depth of the tree would be the length of the largest codeword in the code. ($l_{\text{max}}=max_{x \in \mathcal{X}} l(x)$)\\

Suppose there is a codeword $\underbar{c}$ of length $l$ represented by a node at depth l($l \leq l_{\text{max}}$).\\

Then $\underbar{c}$ has $2^{l_{\text{max}}-l}$ successors at level $l_{\text{max}}$. Also, none of these are codewords as the code is prefix-free. Now,

\begin{align}
    \sum_{\underbar{x} \in \mathcal{X}} 2^{l_{\text{max}}-l(\underbar{x})} \leq 2^{l_{\text{max}}}
\end{align}
This is true if distinct codewords $\underbar{c1}$, $\underbar{c2}$ don't have common successors at $l_{\text{max}}$ level. Any successor of $\underbar{c1}$ at $l_{\text{max}}$ has $\underbar{c1}$ as a prefix. Similarly in the other case as well.

Suppose $l(\underbar{c1}) \leq l(\underbar{c2})$, so there can be a common successor $\underbar{v}$ at $l_{\text{max}}$.

Then, $\underbar{v}$ has first $l(\underbar{c1})$ places as $\underbar{c1}$ and first $l(\underbar{c2})$ places as $\underbar{c2}$.\\

$\Rightarrow$ $\underbar{c1}$ should be a prefix of $\underbar{c2}$ which isn't true as $\mathscr{C}$ is a prefix-free code. Hence, no pair of codewords in $\mathscr{C}$ have any common successors.

Hence $(1)$ is true. And dividing both sides by $2^{l_{\text{max}}}$ gives us the Kraft inequality.

\subsection{Lemma}
$$ L^{*} \geq H(X)$$
Any prefix-free code for $X$ has average length of at least $H(X)$.

Proof:
$$ L - H(X) =  \sum_{x \in \mathcal{X}}p(x)l(x) - \sum_{x \in \mathcal{X}} p(x) \log \frac{1}{p(x)}$$
We know,
 $$ D(p||q)= \sum_{x \in supp(P_X)}p(x)\log \frac{p(x)}{q(x)}$$
 Let
$$ q(x):= \frac{2^{-l(x)}}{\sum_{x \in \mathcal{X}} 2^{-l(x)}}   $$

\begin{align*}
    D(p_X || q_X) &= \sum_{x \in supp(P_X)}p(x)\log \frac{p(x)}{\frac{2^{-l(x)}}{\sum_{x' \in \mathcal{X}} 2^{-l(x')}}} \\
    &= - \sum p(x) \log \frac{1}{p(x)} + \sum p(x) \log \frac{1}{\frac{2^{-l(x)}}{\sum_{x' \in \mathcal{X}} 2^{-l(x')}}} \\
    &= -H(X) + \sum_{x \in supp(p_x)} p(x) \log 2^{l(x)} + \sum_{x \in supp(p_x)} p(x) \log \sum_{x'} 2^{-l(x')}  \\
    &= -H(X)+ L - \epsilon \qquad (\text{from Kraft's})\\
    & \leq - H(X)+L
\end{align*}
$$ \Rightarrow L_{\mathscr{C}} - H(X) \geq 0$$
 $$ L^{*} \geq H(X)$$

 Note: Equality happens only iff $p_x = q_x$ and $\epsilon$ is zero.

\end{document}
