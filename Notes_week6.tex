\documentclass{article}
\usepackage{mathrsfs}
\usepackage{amsmath}
\usepackage{mathtools}
\DeclarePairedDelimiter{\ceil}{\lceil}{\rceil}


\begin{document}
\begin{center}
\textbf{\huge{Week 6}}
\end{center}

\section{Random variable source coding}

Let $\underbar{c}(x)$ be the codeword assigned to $x \in \mathcal{X}$.

$l(x)$ be the length of codeword assigned to $x$.

Here we are coding a single random variable and all codewords are binary strings. (Fixed-variable length source coding)
$$ \mathscr{C}= \{ \underbar{c}(x): x \in \mathcal{X}\}$$

$$ L_{\mathscr{C}}= \sum_{x \in \mathcal{X}}p(x)l(x)$$

Our goal is to design a code which has minimum $L_{\mathscr{C}}$, $L^{*}$
$$L^{*} = \text{min}_{\mathscr{C}} L_{\mathscr{C}}$$

\subsection{Kraft inequality}

Let $\mathscr{C}$ be any prefix-free (binary) code. Then,
$$ \sum_{x \in \mathcal{X}} 2^{-l(x)} \leq 1$$

Proof: We know that any prefix-free code can be represented using a binary tree which has `leaves' as the codeword. In the binary tree corresponding to the code, the depth of the tree would be the length of the largest codeword in the code. ($l_{\text{max}}=max_{x \in \mathcal{X}} l(x)$)\\

Suppose there is a codeword $\underbar{c}$ of length $l$ represented by a node at depth l($l \leq l_{\text{max}}$).\\

Then $\underbar{c}$ has $2^{l_{\text{max}}-l}$ successors at level $l_{\text{max}}$. Also, none of these are codewords as the code is prefix-free. Now,

\begin{align}
    \sum_{\underbar{x} \in \mathcal{X}} 2^{l_{\text{max}}-l(\underbar{x})} \leq 2^{l_{\text{max}}}
\end{align}
This is true if distinct codewords $\underbar{c1}$, $\underbar{c2}$ don't have common successors at $l_{\text{max}}$ level. Any successor of $\underbar{c1}$ at $l_{\text{max}}$ has $\underbar{c1}$ as a prefix. Similarly in the other case as well.

Suppose $l(\underbar{c1}) \leq l(\underbar{c2})$, so there can be a common successor $\underbar{v}$ at $l_{\text{max}}$.

Then, $\underbar{v}$ has first $l(\underbar{c1})$ places as $\underbar{c1}$ and first $l(\underbar{c2})$ places as $\underbar{c2}$.\\

$\Rightarrow$ $\underbar{c1}$ should be a prefix of $\underbar{c2}$ which isn't true as $\mathscr{C}$ is a prefix-free code. Hence, no pair of codewords in $\mathscr{C}$ have any common successors.

Hence $(1)$ is true. And dividing both sides by $2^{l_{\text{max}}}$ gives us the Kraft inequality.

\subsection{Lemma}
$$ L^{*} \geq H(X)$$
Any prefix-free code for $X$ has average length of at least $H(X)$.

Proof:
$$ L - H(X) =  \sum_{x \in \mathcal{X}}p(x)l(x) - \sum_{x \in \mathcal{X}} p(x) \log \frac{1}{p(x)}$$
We know,
 $$ D(p||q)= \sum_{x \in supp(P_X)}p(x)\log \frac{p(x)}{q(x)}$$
 Let
$$ q(x):= \frac{2^{-l(x)}}{\sum_{x \in \mathcal{X}} 2^{-l(x)}}   $$

\begin{align*}
    D(p_X || q_X) &= \sum_{x \in supp(P_X)}p(x)\log \frac{p(x)}{\frac{2^{-l(x)}}{\sum_{x' \in \mathcal{X}} 2^{-l(x')}}} \\
    &= - \sum p(x) \log \frac{1}{p(x)} + \sum p(x) \log \frac{1}{\frac{2^{-l(x)}}{\sum_{x' \in \mathcal{X}} 2^{-l(x')}}} \\
    &= -H(X) + \sum_{x \in supp(p_x)} p(x) \log 2^{l(x)} + \sum_{x \in supp(p_x)} p(x) \log \sum_{x'} 2^{-l(x')}  \\
    &= -H(X)+ L - \epsilon \qquad (\text{from Kraft's})\\
    & \leq - H(X)+L
\end{align*}
$$ \Rightarrow L_{\mathscr{C}} - H(X) \geq 0$$
 $$ L^{*} \geq H(X)$$

 Note: Equality happens only iff $p_x = q_x$ and $\epsilon$ is zero.
 %30/6
 \subsection{Lemma}
 Suppose we have a random variable $X \in \{ x_1, x_2, \cdots, x_k\}$ and positive integers such that $\sum_{i=1}^{k} 2^{-l_i} \leq 1$.

 Then there exists a prefix free code for $X$ with codeword lengths $l_1 , l_2 , \cdots, l_k$.\\

 Proof: We can construct a binary tree with leaves at depths $l_1, l_2, cdots, l_k$ such that none of these nodes are successors of each other, i.e. they are leaves of some binary tree (valid p-f code).

 Assume that $l_1 \leq l_2 \leq \cdots \leq l_k$, without loss in generality    , for any $i \leq k$,
 \begin{equation}
     \sum_{1}^{i-1} 2^{-l_i} <1
 \end{equation}

Imagine we take the full binary tree upto level $l_k$. At each step of the algorithm we intend to pick one available (undeleted) node from the above tree at level $l_i$ and delete all it's successors from the tree. Repeat this process for $i=1, \cdots , k$, we will then have a prefix-free tree.

We have to show that at each step $i=1, \cdots, k$, there is atleast one node left undeleted at depth $i$. We shall use observation $(2)$.

Clearly at step 1, there is a node at $l_1$, ($l_1 \geq 1$).

After $i-1$ steps, assume that we have picked nodes at level $l_1, \cdots, l_{i-1}$ and appropriately deleted. We want to show that there is a node at level $i$.

Total nodes at level $l_k$ which aren't in the tree after $(i-1)^{th}$ step
$$ = \sum_{j=1}^{i-1} 2^{l_k -l_j}$$

Hence, number of nodes remaining at level $l_k$
$$ = 2^{l_k}(1- \sum_{j=1}^{i-1}2^{-l_j})$$

$\Rightarrow$ Number of nodes remaining at $l_k$ in tree at after $(i-1)^{th}$ step $>1$.

$\Rightarrow$ At least one  survivor node should be present at level $l_i$ also. So we can pick a node for the $i^{th}$ step from level $l_i$ also. This completes the proof.

Remark: We construct the tree from smallest length to largest length codewords. Contrast this with optimal source coding we shall see Hoffman codes later.

\subsection{}

Now, suppose that the source random variable $X \sim P_X$, we want to obtain a collection of integers $l_1, \cdots, l_k$ such that Kraft inequality is satisfied, then we know how to get the code for $X$.
$$ \sum_{i=1}^{k} 2^{-l} \leq 1$$
\begin{itemize}
    \item Suppose all codewords are of the same length, $$ k 2^{-l} \leq 1 \Rightarrow l \geq \log k$$

    Hence we can pick $l = \ceil{\log k}$ (ceil function). But there is no guarantee that this code is `good', it may not have small average length.

    \item We know average length $= \sum p_i l_i$.

    Then we will choose small $l_i$ for larger $p_i$. (while taking care that it satisfies Kraft inequality)
\end{itemize}

\section{Shannon-Fano code}

We fix,
\begin{equation}
    l_i = \ceil{ \log_2 \frac{1}{p_i}}
\end{equation}

 where $p_i$ is the probability of $X$ taking the $i^{th}$ value in $\mathcal{X}$.

Clearly $l_i \geq 1$.

\begin{align*}
    \sum_{i=1}^k 2^{-l_i} &= \sum_{i=1}^k 2^{-\ceil{\log_2 \frac{1}{p_i}}}  =\leq \sum_{i=1}^{k} 2^{- \log_2 \frac{1}{p_i}} \\
    &= \sum_{i=1}^k p_i = 1
\end{align*}

The lengths given by $(3)$ satisfy Kraft inequality. We can use the tree-pruning algorithm (see sec 1.3) to get a prefix-free code for $X$.

This code obtained is called as the Shannon-Fano code.

Now,

\begin{align*}
    L_{\text{Shannon-Fano}} &= \sum_{i=1}^{k} p_i \ceil{\log_2 \frac{1}{p_i}} \\
    &< \sum_{i=1}^{k} p_i \left( \log_2 \frac{1}{p_i} +1 \right) \\
    &< H(X)+1
\end{align*}

But S-F code is not always an optimal length prefix-free code.

\subsection{Example for Shannon-Fano code}
$$ X \in \mathcal{X}= \{ x_1,x_2,x_3,x_4 \}$$
\begin{equation*}
        P_X (x_i)=
        \begin{cases}
          1/4, & \text{if}\ i=1 \\
          1/2, & \text{if}\ i=2 \\
          1/9, & \text{if}\ i=3 \\
          5/36, & \text{if}\ i=4
        \end{cases}
    \end{equation*}

Lengths satisying the Kraft's inequality for the Shannon-Fano code
$$ l_i= \ceil{\log_2 \frac{1}{P_X (x_i)}}= \begin{cases}
  2, & \text{if}\ i=1 \\
  1, & \text{if}\ i=2 \\
  4, & \text{if}\ i=3 \\
  3, & \text{if}\ i=4
\end{cases} $$

$$ \bar{L}= \sum_{i=1}^{4} p_i l_i = \frac{57}{36}$$

Obtaining the Shannon-Fano code:
\begin{itemize}
    \item Arrange lengths in ascending order: $l_2 \leq l_1 \leq l_4 \leq l_3$. Hence we pick off a node at depth 2, then 1, 4 and at 3 and delete all the successors.

    \item Now we can choose the codewords from the binary tree, say
    $$ x_1 \to 01 \qquad x_2 \to 1 \qquad x_3 \to 0010 \qquad x_4 \to 000$$
    Hence, the prefix-free Shannon-Fano code is $\{ 01,1,0010,000\} $.
\end{itemize}

\section{Huffman coding}
We will show an optimal code construction that has smaller average length compared to the Shannon-Fano code.

Let us see some intuitive lemma's about an optimal code for a random variable $X$:

Assume that $X \in \mathcal{X}= \{ x_1, \cdots, x_k\}$.

$P_X(x_i)= p_i, i=1, \cdots , k$, without loss in generality, let $p_1 \geq \cdots \geq p_k$.
\begin{enumerate}
    \item Consider that $l_1, \cdots, l_k$ are the lengths of the codewords associated to the messages $x_i, i=1,\cdots, k$ respectively in any optimal code for X.

    Then,
    \begin{equation}
        l_1 \leq \cdots \leq l_k
    \end{equation}

    Proof: Suppose $\mathscr{C}$ is an optimal code in which $\exists $ some distinct $i,j \in \{1,\cdots, k \}$ such that $p_i > p_j$ but $l_i > l_j$. (Assumption of the contrary)

    We shall show that there is another code $\mathscr{C}'$ which has smaller average length $L_{\mathscr{C'}}$ than $L_{\mathscr{C}}$.

    Cosider the code $\mathscr{C}'$ in which the codewords for $x_i$ \& $x_j$ are swapped between those in $\mathscr{C}$.
    $$ \underbar{c}_{\mathscr{C'}}(x_j)=\underbar{c}_{\mathscr{C}}(x_i) \qquad \underbar{c}_{\mathscr{C'}}(x_i)=\underbar{c}_{\mathscr{C}}(x_j)$$

    Now in $\mathscr{C}$,

    $$l_{i, \mathscr{C}} :=  \text{length of codeword of } x_i \text{ in }\mathscr{C}= l_j$$
$$l_{i, \mathscr{C'}}= l_i$$

So, $l_{i, \mathscr{C'}}$ is having the same codewords as $\mathscr{C}$ and hence is also prefix-free.
$$ \bar{L}_\mathscr{C'}= \sum_{k= \{ 1, \cdots, k\} \setminus \{ i,j\}}  p_k l_k+ p_i l_i + p_j l_j$$
$$ \bar{L}_\mathscr{C}= \sum_{k= \{ 1, \cdots, k\}} p_k l_k$$
$$ \bar{L}_\mathscr{C'} - \bar{L}_\mathscr{C} = p_i(l_j -l_i)+ p_j(l_i -l_j) =(p_i-p_j)(l_j -l_i) $$

We have $p_i > p_j$, but $l_i > l_j$,
$$ \Rightarrow  \bar{L}_\mathscr{C'} - \bar{L}_\mathscr{C} < 0 \Rightarrow  \bar{L}_\mathscr{C'} < \bar{L}_\mathscr{C}$$
$\Rightarrow$ $\mathscr{C}$ is not optimal, thus this a contradiction. $(4)$ has to happen for an optimal code.

\item Consider the tree representation of a code which is optimal code. In such a tree, it should be true that either each node is a codeword or it has at least 2 successors which are codewords. i.e. There are no unused leaves in the tree of an optimal code.
\end{enumerate}


\end{document}
